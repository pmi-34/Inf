% Это комментарий
% Комментарии начинаются с символа %
% Формат А4, стиль - отчет
\documentclass[a4paper,12pt]{report}
% Поля - 2 сантиметра с каждой стороны, без места на подшивку
\usepackage[
  left   = 1cm,
  right  = 1cm,
  top    = 1cm,
  bottom = 2cm,
  bindingoffset = 0cm
]{geometry}
% Входная кодировка - UTF-8
\usepackage[utf8]{inputenc}
% Язык текста - русский и английский. Русский - основной
\usepackage[english,russian]{babel}
% Для команды \Box и других математических
\usepackage{amsfonts}
% Исправляем таблицы
\usepackage{float}
\restylefloat{table}
% Для вставки изображений
\usepackage{graphicx}
% Для вставки исходного кода
\usepackage{listings}
% Настройки листингов
\lstloadlanguages{Pascal}
\lstset{
language=Pascal,
extendedchars=\true,
inputencoding=utf8,
commentstyle=\itshape,
stringstyle=\bf,
belowcaptionskip=5pt
}
% Используем графы
\usepackage{tikz}
\usetikzlibrary{snakes,arrows,shapes}
% Первый абзац должен начинаться с красной строки
\usepackage{indentfirst}
% Мы не используем главы, поэтому подправим номера секций
\renewcommand{\thesection}{\arabic{section}.}
% Сокращение для стрелочки
\renewcommand{\a}{\rightarrow}
\newcommand{\A}{\Rightarrow}
\renewcommand{\b}{\fbox}
% Исправим досадную ошибку с неразрывным пробелом в UTF-8
\DeclareUnicodeCharacter{00A0}{~}
% Моя команда для вставки кода
\DeclareRobustCommand{\clist}[1]
{
  \lstinputlisting[mathescape]{#1}
}
% Поправим отступы у списков
% Нумерованные списки в виде а)
\usepackage{enumitem}
\AddEnumerateCounter{\Asbuk}{\@Asbuk}{\CYRM}
\AddEnumerateCounter{\asbuk}{\@asbuk}{\cyrm}
\setlist[enumerate,1]{leftmargin=\dimexpr 26pt-.5in,label=\arabic{enumi}.}
\setlist[enumerate,2]{label=\asbuk{enumii})}
\setlist[enumerate,3]{label=\asbuk{enumiii})}

\begin{document}
\begin{center}
{\large \bf Лабораторная работа \No 4}

{\it Вариант \No 2}
\end{center}
\begin{enumerate}[leftmargin = 10pt]
  \item {\it Определите, какую сумму вычисляет данный алгоритм. Найдите 
        порядок временной сложности данного алгоритма.}
    \clist{4_1.pas}
    Похоже, в задании опечатка: цикл должен начинаться с i = 1; в таком случае
    он будет считать сумму $2! + 4! + \dots + (2n)!$; в текущем виде он 
    считает сумму $\frac{4!}{2} + \frac{6!}{2} + \dots + \frac{(2n)!}{2}$. 
    Сложность приведенного алгоритма равна $2 + (n-1)\times(6 + 2) = 8n - 6 = 
    O(n)$ - линейная.
  \item 
    \begin{enumerate}
      \item {\it Определите, какую сумму вычисляет данный алгоритм. Найдите 
            порядок временной сложности данного алгоритма.}
        \clist{4_2-1.pas}
        Данный алгоритм считает сумму $1 + \frac{2}{4} + \frac{3}{8} + \dots
        + \frac{k}{2^k}$. Порядок его сложности: $1 + (k-1)\times(1 + k\times2
        + 3) = 2k^2 + 2k - 3 = O(k^2)$
      \item {\it Проведите оптимизацию алгоритма, решив эту же задачу другим
            способом, но так, что бы сложность алгоритма уменьшилась на 
            порядок. Укажите порядок временной сложности оптимизированного
            алгоритма.} \\
            Преобразуем нашу сумму, приведя все члены к одному знаменателю:
            $\frac{1}{2} + \frac{1}{2} + \frac{2}{4} + \frac{3}{8} + \dots +
            \frac{k}{2^k} = \frac{1}{2} + \frac{1\times2^{k-1} + 
            2\times2^{k-2} + 3\times2^{k-3} + \dots + k\times2^{k-k}}{2^k} =
            \frac{1}{2} + \frac{1\times2^{k} + 
            2\times2^{k-1} + 3\times2^{k-2} + \dots +
            k\times2^{k-k+1}}{2^{k+1}}$
            \clist{4_2-2.pas}
            Порядок сложности данного алгритма: $1 + k\times5 + 6 = 5k + 7 =
            O(k)$, что ниже, чем $O(k^2)$.
    \end{enumerate}
  \item {\it Даны два алгоритма. Оцените порядок сложности каждого из них.
        Какой из них имеет меньшую сложность?}\\
    \begin{enumerate}
      \item
        \clist{4_3-1.pas}
        Итоговая сложность: $2 + V^2\times(1 + V\times3 + 1 + log_2(log_2V))
        \times(1 + 2 + 3) + 3) + 2 = V^2\times(5 + V\times3 + log_2(log_2(V))
        \times6) + 4 = 3V^3 + 6V^2log_2(log_2(V)) + 5V^2 + 4 = O(V^3)$
      \item
        \clist{4_3-2.pas}
        Итоговая сложность: $2 + V\times(1 + V\times3 + 2 + log_2(V)\times(2 +
        3 + 2) + 3) + 2 = V\times(V\times3 + log_2(V)\times7 + 6) + 4 = 3V^2 +
        7Vlog_2(V) + 6V + 4 = O(V^2)$
    \end{enumerate}
    Очевидно, что второй алгоритм имеет меньшую сложность.
  \item {\it Даны описания типов данных и используемых переменных.}
    \clist{4_4-0.pas}
    {\it Так же даны два алгоритма для вычисления значения полинома n-й 
    степени: \\
    $p(x) = a_nx^n + a_{n-1}x^{n-1} + \dots + a_1x^1 + a_0x^0$
    }\\
    \begin{enumerate}\leftmargin-1pt
    \item {\it Оцените порядок сложности каждого из двух алгоритмов.}
    \begin{enumerate}
    \begin{minipage}{0.5 \textwidth}
    \item
      \clist{4_4-1.pas}
      Итоговая сложность: \\
      $1 + (n+1)\times(1 + n\times2 + 3) = \\ 1 + 4n + 2n^2 +
      4 + 2n = \\ 2n^2 + 6n + 5 = O(n^2)$
    \end{minipage}
    \begin{minipage}{0.5 \textwidth}
    \item
      \clist{4_4-2.pas}
      Итоговая сложность: \\
      $1 + 1 + n\times(2 + 3) = 5n + 2 = O(n)$
    \end{minipage}
    \end{enumerate}
    \item {\it Создайте для решения этой же задачи еще более эффективный 
          алгоритм.}
    \clist{4_4-3.pas}
    Сложность данного алгоритма: $1 + n\times3 = 3n + 1 = O(n)$,
    несколько меньше сложности самого эффективного из этих двух
    алгоритмов, хотя порядок остался тем же.
    \end{enumerate}
  \item {\it Напишите программу для вычисления суммы. Составьте алгоритм 
        вычисления таким образом, чтобы временная сложность была 
        $O(nlog_2(n))$:} \\
        $S = a^1 + (a+1)^2 + (a+2)^3 + \dots + (a+n-1)^n$
    \clist{4_5.pas}
    Откуда взялась логарифмическая сложность функции? Нетрудно заметить из
    реализации, что общее количество операций не превосходит количества цифр 
    в двоичной записи числа $b$. Оценивая сверху следующей степенью двойки,
    такой, что $2^n \ge b$, получаем требуемое число операций $n = log_2(b)$.
\end{enumerate}
\end{document}
